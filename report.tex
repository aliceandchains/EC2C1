\documentclass{article}
\usepackage{graphicx}
\usepackage{booktabs}
\usepackage{multirow}
\usepackage{adjustbox}
\usepackage[a4paper, margin=1in]{geometry}
\usepackage{graphicx}
\usepackage{caption}
\usepackage{subcaption}
\usepackage{float}  % Optional, for [H] positioning

\title{Wings vs Wells. Assessing causal impacts of Energy Infrastructure on Avian Populations: A replication and extension of Katovich (2024)}
\author{EC2C1}
\date{May 2025}

\begin{document}

\maketitle

\section{Introduction}

This paper replicates and extends the econometric methodology published by Katovich (2024), who investigates the casual effect of expansion in energy infrastructure, specifically wind turbines and gas shale wells, on bird populations varied across species and types of habitats. The original work proposes a population-level approach to evaluating the impact by mapping longitudinal micro-data from the National Audubon Society’s Christmas Bird Count with geolocated registries of all wind turbines and shale wells constructed for the entire lower 48 United States between 2000 and 2020. Such a methodology improves prior research such as (Barton et al., 2016) that heavily relied on extrapolation tactics which typically lack external validity due to selection bias when dealing with site-specific investigations.
 
\addlinespace

Katovich (2024) uses bivariate DiD and Poisson continuous settings to estimate both the treatment and marginal effects of infrastructure installation. The bivariate DiD model uses the \textit{csdid} estimator that appropriately weighs group-time comparisons to avoid bias when comparing treated circles with those that have already been treated, which is important to account for any heterogeneity caused by the difference in time of infrastructure arrival. In addition to the bivariate model that tracks only the first installment of the infrastructure, the continuous setting further investigates the effect by capturing the intensity of treatment by considering the cumulative number of turbines or shale wells in a given circle. The core regression incorporates circle fixed effects, controlling for unobserved, time-invariant differences across circles, such as long-standing regulatory environments, as well as year fixed effects to absorb common shocks affecting all circles in a given year such as federal environmental policies that may be critical in determining the installation of energy infrastructure. Standard errors are clustered at the circle level to account for the autocorrelation between error terms and heteroskedasticity within each circle over time which can persist after controlling for weather, land conditions, and counting efforts on the date of data collection. For instance, bird disease in a particular location in a given year will likely affect bird count in the subsequent periods. Although clustering on the circle level does not account for potential spillover effects between neighboring treated and untreated circles, Katovich additionally employs the Correlated Spatial Random Effects model, allowing spatial lags in dependent variables and spatially lagged errors. Lastly, the paper investigates whether the human population acts as a mediator, testing if the energy infrastructure effects bird counts through the inflow of the population into areas. 

\addlinespace

Overall, core findings conclude that shale wells have a statistically significant negative effect on bird populations, whereas wind turbines showed no meaningful effect. The same result is shared by continuous and CSRE models. Similarly, the human population has not appeared as a mediator, supporting claims about direct casual effects found earlier.

\section{Replication}

\addlinespace
\[
y_{ct} = \beta_1 T_{ct} + \mathbf{X}_{ct}' \boldsymbol{\beta}_2 + \gamma_c + \delta_t + \varepsilon_{ct} \hspace{2em} (1)
\]
\addlinespace

The dataset used for replication is close to the one used by Katovich (2024) and includes \textit{31,000+} rows of panel data. Prior to the analysis, all data was winsorised at the \textit{99th} percentile to exclude the impact of any extremes. Instead of the \textit{csdid} estimator, this report uses a standard Two-Way Fixed-Effects DiD model in combination with circle and time fixed effects. Equation (1) specifies the regression where the outcome of interest \( y_{ct} \) is the \textit{asinh()} transformed bird population or species count (to handle zero and extreme values), and \( T_{ct} \) is the binary treatment variable indicating whether a wind turbine or shale well was installed in a particular circle. In this setting, \( \beta_1 \) estimates an approximate \textit{\%} change in the outcome (semi-elasticity), given that the treatment occurs. The \( X_{ct} \) matrix of covariates includes parameters for controlling a number of cofounding sources that can bias the estimate due to the effect on both infrastructure placement and bird outcomes. For example, (i) weather conditions (max and min temperature, snow, and wind speed); and (ii) land-use composition within the circle (the proportion of agricultural, pasture, and developed land). Fixed effects control for any unobserved time-invariant cofounders as outlined earlier. It is important to note that the model relies on a parallel trend assumption, which assumes that in the absence of treatment, both treated and untreated circles would follow the same trend in the development of bird and species counts. Robust standard errors are clustered at the circle level as discussed before.

\addlinespace

Tables 1 and 2 display estimated effects of wind turbine and shale well installations on bird populations and species, respectively. Following the methodology of Katovich (2024), the same set of covariates was used to enhance comparability. The results in the tables closely replicate the outcomes of the original paper. 

\begin{table}[!ht]
\centering
\caption{Estimated effect from Turbines on Bird Population}
\begin{adjustbox}{width=\textwidth}
\begin{tabular}{@{}lcccccccccc@{}}
\toprule
 & Total & Grassland & Woodland & Wetland & Other Habitat & Urban & Non-Urban & Residents & Short Mig. & Longer Mig. \\
\midrule
\addlinespace
\multicolumn{11}{l}{\hspace{-0.5em}\textbf{Total count}} \\
Coef. & 0.005 & 0.042 & 0.082 & 0.016 & -0.018 & 0.000 & 0.034 & 0.015 & -0.008 & 0.011 \\
St. Error & (0.032) & (0.030) & (0.023) & (0.055) & (0.039) & (0.033) & (0.037) & (0.024) & (0.034) & (0.050) \\
p-val & 0.887 & 0.160 & 0.000 & 0.764 & 0.649 & 0.994 & 0.357 & 0.548 & 0.819 & 0.829 \\
\addlinespace
\multicolumn{11}{l}{\hspace{-0.5em}\textbf{Total species}} \\
Coef. & 0.008 & 0.014 & 0.016 & -0.003 & -0.001 & 0.008 & 0.007 & 0.012 & 0.004 & 0.009 \\
St. Error & (0.008) & (0.012) & (0.008) & (0.018) & (0.008) & (0.006) & (0.012) & (0.007) & (0.009) & (0.017) \\
p-val & 0.315 & 0.249 & 0.035 & 0.842 & 0.905 & 0.181 & 0.569 & 0.063 & 0.668 & 0.580 \\
\bottomrule
\end{tabular}
\end{adjustbox}
\end{table}

\addlinespace

The effect of wind infrastructure appears to have no statistical significance for the aggregated bird population or species count with near-zero, yet positive coefficients of \textit{(0.005) and (0.008)} as well as insignificant \textit{p-values = (0.887) and (0.315)}. The consistently positive coefficient might be the result of the omitted variable bias (OVB) related to unobserved factors such as local biological conservation policies in circles that attract wind turbines and independently raise bird counts. On the disaggregated level, however, woodland species experienced a positive significant effect \textit{p-values = (0.000) and (0.036)}, implying wind turbine installation increases woodland bird counts and species by \textit{8.2\%} and \textit{1.6\%} accordingly, which might sound counterintuitively but can suggest a reasonable ecological benefit that fits into the wider environment for these species through behavioural shift or simply an (OVB) in the absence of site-specific unaccounted characteristics.

\addlinespace

On the opposite side, results for shale wells present a more grounded argument for their statistical significance in negatively impacting the bird count, which is especially notable at a disaggregated level with counts for grassland and longer migration species declining by \textit{23.2\%} and \textit{23.1\%} with \textit{p-values = (0.001) and (0.013)}. While estimated coefficients are consistently negative and align with Katovich's (2024) results, many of them lack statistical validation which is observable on the aggregated level with estimated declines in population and species counts of \textit{12.8\%} and \textit{1.7\%} with \textit{p-values = (0.120) and (0.363)}. While the original paper finds significance, it's likely attributed to the difference in estimators used - more robust \textit{csdid vs TWFE} and dataset. Although not yet statistically significant, the suggested economic significance is substantial. Given the average untreated circle has \textit{18,338} birds, this estimates a reduction of 2,347 birds per treated circle, showcasing a devastating ecological impact.

\begin{table}[!ht]
\centering
\caption{Estimated effect from Shale wells on Bird Population}
\begin{adjustbox}{width=\textwidth}
\begin{tabular}{@{}lcccccccccc@{}}
\toprule
 & Total & Grassland & Woodland & Wetland & Other Habitat & Urban & Non-Urban & Residents & Short Mig. & Longer Mig. \\
\midrule
\addlinespace
\multicolumn{11}{l}{\hspace{-0.5em}\textbf{Total count}} \\
Coef. & -0.128 & -0.232 & -0.071 & -0.069 & -0.064 & -0.110 & -0.158 & -0.146 & -0.104 & -0.231 \\
St. Error & (0.082) & (0.069) & (0.065) & (0.107) & (0.109) & (0.092) & (0.081) & (0.051) & (0.093) & (0.093) \\
p-val & 0.120 & 0.001 & 0.277 & 0.518 & 0.559 & 0.234 & 0.052 & 0.005 & 0.262 & 0.013 \\
\addlinespace
\multicolumn{11}{l}{\hspace{-0.5em}\textbf{Total species}} \\
Coef. & -0.017 & -0.048 & -0.005 & 0.020 & 0.010 & -0.005 & -0.028 & -0.028 & -0.001 & -0.061 \\
St. Error & (0.018) & (0.027) & (0.016) & (0.047) & (0.018) & (0.014) & (0.029) & (0.013) & (0.021) & (0.037) \\
p-val & 0.363 & 0.079 & 0.749 & 0.677 & 0.579 & 0.706 & 0.335 & 0.030 & 0.963 & 0.096 \\
\bottomrule
\end{tabular}
\end{adjustbox}
\end{table}

\addlinespace

To further reinforce replication, a wider spectrum of controls was used to better capture the difference in geographical data beyond the original specification. For example, covariates for capturing the form of still and moving water (open/frozen) are relevant controls since they are exogenous to treatment and could present vital impacts on some types of birds, especially those with the wetland form of habitat, therefore reducing the (OVB). Running this extended form produced results close to the earlier findings with only small variations in estimates and significance, hence, serving as a robustness check.

\section{Extended Analysis}

As previously discussed, parallel trends lie at the core of a standard Two-Way Fixed-Effects DiD model, however, with an extensive and not perfectly balanced panel dataset with over \textit{2000} circles and up to 20 periods, this assumption becomes harder to sustain due to the higher likelihood of heterogeneity in pre-treatment periods. Figure 1 plots average \textit{asinh()} transformed Bird counts for each group against years before and after pseudo treatment year - 2010, which is positioned in the middle of the initial turbine and shale well installation distribution, to visualise trends for wind turbines and shale wells. The graphs suggest mostly parallel trends for control and treated groups, although, in the case of wind turbines the treated group appears to have a higher bird count for most of the period. To further check results, the previous DiD model is adjusted by relaxing the parallel trends assumption for each circle.

\addlinespace
\[
y_{ct} = \beta_1 T_{ct} + \mathbf{X}_{ct}' \boldsymbol{\beta}_2 + \gamma_c + \delta_t + \lambda_c \cdot t + \varepsilon_{ct} \hspace{2em} (1')
\]
\addlinespace

Regression (1') adds an interaction term \( \lambda_c \cdot t \), allowing each circle to follow its own trend, therefore enabling \( \beta_1 \) to absorb more variation attributable to treatment and avoiding bias arising due to unique trends across circles. Table 3 presents a summary of estimates for aggregated levels of data. It's worth noting that several observations were omitted due to multicollinearity which reduces accuracy and inflates standard errors.

\addlinespace

For shale wells, the estimated effect remained nearly identical to the results in Table 2, predicting a \textit{12.6\%} reduction in bird count with a \textit{p-value = (0.116)}.The estimates for turbines appeared more negative than those in Table 1, suggesting a \textit{7.2\%} decline in bird count, although the effect remained statistically insignificant with a \textit{p-value = (0.111)}. Overall, the results from the interaction model and the visualised trends in Figure 1 support the earlier findings. For shale wells, both the regression and graphical evidence point to a substantial and nearly significant negative effect. For turbines, however, the evidence is weaker and somewhat conflicting. Taken together with the results from Table 1, there is insufficient evidence to reject the null hypothesis that wind turbine installation has no impact on aggregate bird counts.

\begin{center}
    \text{Figure 1}
\end{center}

\begin{figure}[H]
    \centering
    \begin{subfigure}[t]{0.49\textwidth}
        \centering
        \includegraphics[width=\linewidth]{Turbines.png}
    \end{subfigure}
    \hspace{0pt} % <-- remove extra space
    \begin{subfigure}[t]{0.49\textwidth}
        \centering
        \includegraphics[width=\linewidth]{Shale Wells.png}
    \end{subfigure}
\end{figure}

\addlinespace

The effect on species is minor and not statistically significant similar to the baseline model (1). Applying regression (1') estimated a \textit{0.5\%} reduction in bird species due to Turbine installation with a \textit{p-value = (0.585)}. Likewise, the estimated effect on species count for shale wells is almost identical to Table 2 with a \textit{1.5\%} fall in bird species and \textit{p-value = (0.321)}, underlining the suitability of parallel trends assumptions and a more considerable, yet insignificant, negative impact from shale wells.

\addlinespace

Lastly, to assess the robustness of estimated effects, a Placebo test was conducted with 100 random assignments of treatment to circles that were never actually treated. The distribution of placebo coefficients which is centered around zero under the null of zero treatment effect contrasts with a \textit{-12.6\%} true estimated effect of shale wells on the bird population. Such a result lies far beyond the range of placebo distribution, reflecting a likely causal negative impact of shale well installation on bird counts.

\addlinespace

\begin{table}[htbp]
\centering
\caption{Estimated Effect from Interaction Model on Bird Population}
\begin{tabular}{lcc}
\toprule
 & \textbf{Turbines} & \textbf{Shale Wells} \\
\midrule
\multicolumn{3}{l}{\textbf{Total Count}} \\
Coefficient    & -0.072 & -0.126 \\
Std. Error     & (0.045) & (0.080) \\
p-value        & 0.111 & 0.116 \\
\addlinespace
\multicolumn{3}{l}{\textbf{Total Species}} \\
Coefficient    & -0.005 & -0.015 \\
Std. Error     & (0.008) & (0.015) \\
p-value        & 0.585 & 0.321 \\
\bottomrule
\end{tabular}
\end{table}












\section{Lagged Effects}



The more significant effect of shale wells compared to turbines is additionally supported by extending the analysis by splitting the post-treatment period into separate intervals, covering \textit{0-1, 2-3, 4-5, 6-7} and \textit{8+} years after the first installation. Compared to the original paper and basic DiD model, this setting helps to better investigate any possible lag in the effect, as well as how treatment effects evolved over time. This is a crucial aspect to include since some circles contain only a very short post-treatment observation window whilst others have more than 8 years of post-treatment data. A standard approach would only uncover the average effect, hiding any heterogeneity in patterns of treatment, as well as any lag between installation and the effect on outcome.

\addlinespace

\begin{table}[htbp]
\centering
\caption{Estimated Effect by Post-Treatment Time Window}
\label{tab:window_effects_shale_turbine}
\begin{tabular}{lccccc}
\toprule
 & 0--1 yrs & 2--3 yrs & 4--5 yrs & 6--7 yrs & 8+ yrs \\
\midrule
\textbf{Turbine on Bird Count} \\
Coef.      & -0.065 & 0.010  & 0.051  & 0.010  & 0.053 \\
St. Error  & (0.034) & (0.42) & (0.046) & (0.045) & (0.039) \\
p-val      & 0.054 & 0.803 & 0.262 & 0.833 & 0.181 \\
\midrule
\textbf{Turbine on Species Count} \\
Coef.      & -0.009 & 0.015  & 0.007  & 0.009 & 0.025 \\
St. Error  & (0.008) & (0.009) & (0.011) & (0.012) & (0.011) \\
p-val      & 0.267 & 0.115 & 0.551 & 0.452 & 0.019 \\
\midrule
\textbf{Shale well on Bird Count} \\
Coef.      & -0.057 & -0.160 & -0.087 & -0.159 & -0.183 \\
St. Error  & (0.088) & (0.088) & (0.098) & (0.094) & (0.096) \\
p-val      & 0.518 & 0.070 & 0.375 & 0.092 & 0.057 \\
\midrule
\textbf{Shale well on Species Count} \\
Coef.      & -0.005 & -0.031 & -0.04 & -0.017 & -0.032 \\
St. Error  & (0.018) & (0.021) & (0.023) & (0.022) & (0.024) \\
p-val      & 0.771 & 0.126 & 0.847 & 0.448 & 0.175 \\
\bottomrule
\end{tabular}
\vspace{1mm}
\begin{flushleft}
\footnotesize
\end{flushleft}
\end{table}

The regression (1) was modified by incorporating \textit{5} additional mutually exclusive binary variables, specified for each post-treatment interval. The combination of these provides an interpretation of the marginal treatment effect in each interval. For example, the coefficient for \textit{0-1 yrs}, would estimate the effect on bird/species count after the first \textit{2} years since installation.

\addlinespace

Table 4 uncovered the divergent evolution of treatment effects for turbines and shale wells. Circles with wind turbines installed saw a statistically significant \textit{6.5\%} fall in bird counts during the first two years after installation, while the effect for subsequent windows remained insignificant and slightly positive. For circles treated with shale wells, trends suggest the growing negative effect that becomes more significant over time with an estimated \textit{15.9\%} fall over the first \textit{6-7} years and a \textit{18.3\%} reduction in the bird population after \textit{8+} years of post-treatment. While not all of these findings are statistically significant at \textit{5\%} significance level, they reflect a potential heterogeneity not only in the timing of the effect across various post-treatment periods but also in the pattern of outcome changes for two distinct types of infrastructure. Notably, since the average post-treatment period for turbines is \textit{6.1} years and \textit{4.3} years for shale wells, the standard approach places more weight on the medium-term, without evaluating short-term and long-term effects that play a larger role in decision making and assessing the economic significance of energy projects. 



\section{Human Population as a mediator}

Another concern addressed by the original study is whether a human population acts as a mediator in affecting the bird count. This argument is motivated by the evidence from Wilson (2022), suggesting that increased fracking activity causes inward migration. If this holds, without controlling for the mediation effect, treatment effects would be biased as they absorb part of human influence, potentially overstating the true value. To assess whether the human population is a mediator, two causal relationships need to be established - firstly, the causal effect of turbine/well infrastructure on the population and, secondly, the causal effect of the human population on the bird count. 

\addlinespace

This report uses both bivariate and continuous DiD settings for estimating the effect of energy infrastructure on the local human population, motivated by the idea that larger multi-plant energy projects are more likely to cause a meaningful change in the population level. For both models, the regression (1) was used, controlling for cofounders as discussed in earlier sections. Important to note, that such methodology lacks precision since the human population is given on the county level, while data for energy plants and bird populations are on the circle level. Furthermore, in the absence of controls such as the distance between human habitats and circles, this mismatch can lead to significant bias and measurement errors, undermining the true estimated effect. Additionally, an inflow of population into the state could have arisen due to wider economic policies such as varying tax reforms that can also serve as a cofounder if affects installation of energy projects.

\begin{table}[H]
\centering
\caption{Estimated Effects of Infrastructure on Human Population}
\begin{tabular}{lcc}
\toprule
 & \textbf{Turbines} & \textbf{Shale Wells} \\
\midrule
\textbf{Binary Treatment} \\
Coefficient & -0.015 & -0.021 \\
Std. Error  & (0.006) & (0.015) \\
p-value     & 0.009 & 0.164 \\
\addlinespace
\textbf{Continuous Treatment} \\
Coefficient & -0.000 & 0.000 \\
Std. Error  & (0.000) & (0.000) \\
p-value     & 0.002 & 0.750 \\
\bottomrule
\end{tabular}
\end{table}


\addlinespace

Results for the bivariate model in Table 5 closely follow the findings of the original paper. The installation of the wind turbine is associated with a \textit{1.5\%} fall in the human population whilst the effect from shale wells is negative but statistically insignificant. For continuous settings, the effect is essentially zero which might flag the earlier spatial mismatch issue that causes attenuation bias. However, the negative coefficient found in the bivariate model might indicate the reverse causality problem, where in reality population trends affect the placement of energy infrastructure. For instance, turbines are likely to be built in areas where the human population was declining anyway. If this is true, then treatment becomes endogenous as it is affected by the outcome. Instrumental variables such as the implementation of Renewable Portfolio Standards (dummy variable) by counties can fix the issue as these policies influence infrastructure and are exogenous to migration patterns. 

\begin{table}[H]
\centering
\caption{Estimated Effect of Population on Birds}
\begin{tabular}{lcc}
\toprule
 & \textbf{Total Count} & \textbf{Total Species} \\
\midrule
Coefficient & -0.063 & 0.002 \\
Std. Error  & (0.099) & (0.026) \\
p-value     & 0.525 & 0.939 \\
\bottomrule
\end{tabular}
\end{table}


\addlinespace

Table 6 overviews estimates tracking the link between human population and bird counts. Results indicate that the relationship between the two is insignificant, hence ruling out the possibility of the human population acting as a mediator. Although the human population has a potential relationship with turbines/wells arrival, it's excluded from the set of covariates \( X_{ct} \) in regression (1), hence intentionally avoiding the issue of endogeneity when estimating the core effect of energy infrastructure on birds. However, if the human population actually plays the role of the cofounder, affecting both the bird count and energy plants, then the inability to control for it presents a constraint, undermining precise estimation due to (OVB). While vast insignificance reduces concerns, a more in-depth circle-level investigation of the human population would be beneficial.


\end{document}
